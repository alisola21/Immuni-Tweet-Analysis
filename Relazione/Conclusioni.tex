\chapter*{Conclusioni}

L'analisi sin qui condotta ha portato una rivelazione di risultati circa i temi, i sentimenti e in generale l'attitudine degli utenti nei confronti dell'App Immuni, che nonostante un'iniziale ascesa si è ben presto mostrata come un fallimento tecnologico, non riuscendo a mantenere fede alle promesse fatte dalle istituzioni. 

Tuttavia, dall'analisi dei risultati emerge non tanto un problema tecnologico, quanto un problema di natura organizzativa che coinvolge quindi le cariche istituzionali e non gli sviluppatori dell'applicazione, che comunque hanno realizzato uno strumento funzionante sebbene con alcuni \textit{bug} da risolvere. 
Questa mancata organizzazione da parte dell'allora forza governativa si riscontra soprattutto nella scarsa informazione dell'applicazione verso gli utenti e nella poca chiarezza riguardo il suo funzionamento e alla tipologia di dati che l'utente cede all'app, due aspetti che hanno contribuito nel tempo a sviluppare un vero e proprio caso di \textit{infodemia}\footnote{Eccessiva circolazione di informazioni, talvolta non ufficiali e certificate, che rendono difficile  
l'orientarsi verso un argomento per la mancanza di fonti attendibili (Fonte: Enciclopedia Treccani)}, che ha contribuito a suscitare negli utenti sentimenti di rabbia e paura e, in generale, un forte scetticismo nei confronti di Immuni che, infatti, nel corso del tempo è lentamente caduta in disuso durante l'anno del 2021 e attualmente è quasi del tutto inutilizzata

Un ulteriore aspetto che emerge da questa analisi è come nel corso del tempo il \textit{focus} dell'applicazione si sia spostato dapprima verso il Coronavirus in sé e il contenimento dei contagi attraverso misure restrittive come la quarantena, per poi dirigersi verso temi correlati, invece, a due importanti fenomeni del 2021: l'inizio della campagna vaccinale e l'introduzione del Green Pass, i quali hanno forse ulteriormente contribuito al fallimento dell'applicazione in quanto si sono dimostrati due strumenti capaci di contenere in maniera concreta la diffusione del Coronavirus.

La seguente analisi potrebbe ovviamente essere estesa analizzando anche altri aspetti dei tweet come, ad esempio, la profiliazione degli utenti stessi, analizzando la loro ideologia politica, la loro opinione riguardante il virus per capire se in qualche modo questi aspetti influiscano sull'opinione che essi hanno nei confronti di Immuni 
Inoltre, si potrebbero anche analizzare gli stessi aspetti modificando il canale di raccolta e utilizzando anche i dati raccolti da altre piattaforme social per osservare se le opinioni degli utenti mutino o meno.
