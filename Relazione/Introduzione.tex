\chapter*{Introduzione}
Negli ultimi due anni, la diffusione del \textit{Covid-19} nel mondo ha radicalmente cambiato la società mondiale, limitando i contatti e costringendo le persone a restare in casa con la paura di poter essere contagiate e contagiose e contribuire alla diffusione di questo virus allora sconosciuto.

Per questi motivi, soprattutto nella prima ondata di diffusione del virus, in Italia sono state studiate varie soluzioni per monitorare e contenere i contagi, tra cui l'applicazione \textbf{Immuni}, sviluppata dalla società \textit{Bending Spoons}\footnote{\url{https://bendingspoons.com/index.html}} vincitrice del bando indetto dalla task force incaricata dall’allora ministro dell’Innovazione Paola Pisano. L'applicazione è stata rilasciata per la prima volta in quattro regioni italiane per poi diventare disponibile in tutto il paese dal 15 giugno 2020. \\
L'applicazione, grazie alla funzionalità di \textit{contact tracing} basata sulla geo localizzazione dell'utente e l'attivazione della tecnologia Bluetooth, permette agli utenti di segnalare la loro positività, allertando in modo anonimo le persone con cui erano state a contatto e che potevano essere state contagiate, così da permettere a queste persone eventuali verifiche cliniche e in generale evitare di contagiare altri, contribuendo a ridurre la diffusione del coronavirus. %dava all'utente la possibilità di segnalare la propria positività, la quale veniva poi comunicata a tutti gli altri utenti che potevano così isolarsi per evitare altri contatti e possibili infezioni. In questo modo era possibile tener traccia di eventuali focolai ed evitare ulteriori infezioni.
Tuttavia, nonostante un'iniziale accoglienza da parte dei cittadini, ben presto l'applicazione si è rivelata fallimentare sotto molti punti di vista, \textit{in primis} per problemi di natura organizzativa, in quanto nonostante i nobili intenti dell'applicazione essa non è stata abbastanza pubblicizzata e spiegata in maniera chiara agli utenti che spesso si sono rifiutati di scaricarla per problemi legati ai dati sensibili (come la localizzazione) %che non volevano cedere alla suddetta applicazione, 
nonostante venisse garantito l'anonimato degli utenti che segnalavano la loro positività al Coronavirus.

Tutto questo malessere nei confronti di questa applicazione si è riversato anche sui social, che negli ultimi anni sono diventati il mezzo preferenziale per esprimere opinioni e raccontare esperienza e lanciare provocazioni. Così ben presto, l'applicazione è divenuta del tutto inutilizzata e spesso disprezzata dagli utenti che sui social hanno spesso raccontato la loro esperienza negativa a seguito dell'utilizzo dell'app.\\

Ed è proprio dalle opinioni degli utenti che cercheremo in questa relazione di analizzare perché questa applicazione, nata per aiutare i cittadini italiani e per semplificare il monitoraggio della diffusione del virus, si è rivelata un fallimento, rendendo in questo caso la tecnologia addirittura uno svantaggio per chi la utilizza, non rispondendo quindi alle esigenze degli utenti. Tutta l'analisi sarà incentrata sulle tracce che gli utenti hanno lasciato sui \textit{social network}, in particolare sull'accesa discussione avvenuta su \textit{Twitter}, raccogliendo i \textit{Tweet} scritti nel primo anno di nascita di Immuni cercando di estrarre, grazie all'utilizzo di alcune tecniche di \textit{Machine Learning} per l'analisi automatica del testo, come quelle illustrate nel seminario del Dott. Sebastini dell' ISTI-CNR, possibili indizi che ci aiutino a capire il perché di questo fallimento.

Nella prima parte della relazione verrà illustrato il processo di creazione del \textit{corpus} di dati su cui avverranno le analisi successive, partendo dalla raccolta dei tweet sulla base di alcuni \textit{hashtags} utilizzati per discutere del tema Immuni, per poi passare alla comprensione dei dati e pulizia da eventuali dati errati e/o inutili al fine dell'analisi.

Su questo set di dati verranno poi fatte alcune analisi testuali grazie all'ausilio di tecniche di NLP, utilizzando librerie come NLTK e strumenti come il \textit{Topic Modeling} e la \textit{Sentiment Analysis} per cercare di comprendere, a partire dalle opinioni degli utenti i sentimenti e gli argomenti di maggiore importanza per gli utenti, i possibili problemi legati all'utilizzo di questa applicazione.

%Infine, basandosi sui risultati raccolti, si cercherà di implementare una possibile soluzione per fare un \textit{re-design} dell'app Immuni, utilizzando una apposita tecnica di progettazione, la tecnica \textbf{Scamper}.


%Per capire meglio il perché di questo fallimento si è scelto di basarci sulle opinioni dei cittadini italiani, monitorando i tweet scritti sul tema per un anno



